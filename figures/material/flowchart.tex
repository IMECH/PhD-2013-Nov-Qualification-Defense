\usetikzlibrary{shapes,arrows}
\tikzstyle{decision} = [diamond, draw, fill=blue!20, 
    text width=45, text badly centered, node distance=75, inner sep=0pt]
\tikzstyle{block} = [rectangle, draw, fill=blue!20, 
    text width=50, text centered, rounded corners, minimum height=30]
\tikzstyle{line} = [draw, -latex', very thick]
\tikzstyle{cloud} = [draw, ellipse,fill=red!20, node distance=75,
    text width=30, minimum height=10]
{\footnotesize
\begin{tikzpicture}[scale=0.76]
    % Place nodes
    \node [block](A) at (0,0) {研究内容};
    \node [block](B) at (3,0) {应用};
    \node [block](C) at (3,3.5) {方法};
    \node [block](D) at (3,-3.5) {并行};

    \node [block](E) at (6,1.5) {高分子};
    \node [block](F) at (6,-1.5) {红细胞};
    \node [block](G) at (6,3.5) {\textcolor{gray}{MD, DPD, SPH藕合}};
    \node [block](H) at (6,-3.5) {CPU};

    \node [block](I) at (9,1.5) {\textcolor{gray}{多物理场}};
    \node [block](J) at (9,-1.5) {\textcolor{gray}{乳腺癌细胞}};
    \node [block](K) at (9,-3.5) {\textcolor{gray}{GPU}};

    \node [block](L) at (12,0) {\textcolor{gray}{高性能大规模计算}};

    \node [cloud](M) at (3,1.75) {\textcolor{gray}{势函数}};

    \path [line] (A) -- (B);
    \path [line] (A) |- (C);
    \path [line] (A) |- (D);
    \path [line] (B) -| (E);
    \path [line] (B) -| (F);
    \path [line] (D) -- (H);
    \path [gray, line] (C) -- (G);

    \path [line] (E) -- (I);
    \path [line] (F) -- (J);
    \path [gray,line] (H) -- (K);

    \path [gray, line] (K) -| (L);
    \path [gray, line] (G) -| (L);
    \path [gray, line] (L) -| (I);
    \path [gray, line] (L) -| (J);

    \path [gray, line] (C) -- (M);
    \path [gray, line] (M) -- (B);

\end{tikzpicture}}
